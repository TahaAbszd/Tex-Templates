\documentclass[a4paper,12pt]{article}
\usepackage{graphicx}
\usepackage{fancyhdr}
\usepackage{xcolor}
\usepackage{lipsum}
\usepackage{tocloft}
\usepackage{geometry}
\usepackage{titlesec}
\usepackage{setspace}
\usepackage{xepersian}


\geometry{top=3cm, bottom=3cm, left=3.5cm, right=3.5cm}
\settextfont{Vazir Light}
\definecolor{darkblue}{RGB}{0, 51, 102}

\setstretch{1.5}

\pagestyle{fancy}
\fancyhf{}
\renewcommand{\headrulewidth}{1.5pt}
\renewcommand{\headrule}{\color{darkblue}\hrule}
\fancyhead[R]{\color{darkblue}\textbf{ریاضیات گسسته}}
\fancyhead[L]{\color{darkblue}\textbf{سیاست ها}}

\fancyfoot[C]{\color{gray}\thepage}

\thispagestyle{empty}

\titleformat{\section}{\Large\bfseries\color{darkblue}}{\thesection}{1em}{}
\titlespacing{\section}{0pt}{12pt}{6pt}

\renewcommand{\cftsecleader}{\cftdotfill{\cftdotsep}}
\renewcommand{\cftsecfont}{\color{darkblue}}
\renewcommand{\cftsecpagefont}{\color{darkblue}}

% ------------------------------
\begin{document}
	
	\begin{center}
		\vspace*{4cm}
		\includegraphics[width=0.25\textwidth]{logo.png} \\[1cm]
		\textcolor{darkblue}{\large \textbf{سیاست های درس}} \\
		\textcolor{darkblue}{ریاضیات گسسته} \\
		\textcolor{darkblue}{پاییز ۱۴۰۴}



	\end{center}
	
	\clearpage
	\thispagestyle{empty}
	\begin{center}
		\vspace*{1.5cm}
		{\LARGE \textbf{فهرست مطالب}} \\[0.5cm]
	\end{center}
	\vspace{0.5cm}
	\tableofcontents
	
	\clearpage
	\section{مقدمه}
	{\color{black}
		این بخش به بررسی کلی مفاهیم اولیه ریاضیات گسسته
		\lipsum[1][1-3]
	}
	
	\section{اهداف درس}
	{\color{black}
		هدف این درس، آموزش مفاهیم پایه‌ای و پیشرفته در ریاضیات گسسته ‌است.
		\lipsum[2][1-3]
	}
	
	\section{قوانین و ضوابط}
	{\color{black}
		شامل قوانین مربوط به حضور، امتحانات، پروژه‌ها و نحوه ارزیابی.
		\lipsum[3][1-3]
	}
	
	\section{برنامه درسی}
	{\color{black}
		برنامه هفتگی و سرفصل‌های درسی به تفکیک جلسات در این بخش ارائه می‌شود.
		\lipsum[4][1-3]
	}
	
	\section{منابع}
	{\color{black}
		کتاب‌ها و مقالات پیشنهادی جهت مطالعه بیشتر.
		\lipsum[5][1-3]
	}
	
\end{document}
